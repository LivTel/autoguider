\documentclass[10pt,a4paper]{article}
\pagestyle{plain}
\textwidth 16cm
\textheight 21cm
\oddsidemargin -0.5cm
\topmargin 0cm

\title{IUCAA autoguider Installation}
\author{C. J. Mottram}
\date{}
\begin{document}
\pagenumbering{arabic}
\thispagestyle{empty}
\maketitle
\begin{abstract}
This document describes the installation procedure for the IUCAA autoguider.
\end{abstract}
\centerline{\Large History}
\begin{center}
\begin{tabular}{|l|l|l|p{15em}|}
\hline
{\bf Version} & {\bf Author} & {\bf Date} & {\bf Notes} \\
\hline
0.1 & C. J. Mottram & 14/01/13 & First draft \\
\hline
\end{tabular}
\end{center}

\newpage
\tableofcontents
\listoffigures
\listoftables
\newpage

\newcommand{\mytilde}{\raise.17ex\hbox{$\scriptstyle\mathtt{\sim}$}}

\section{Introduction}

This document describes how to unpack and check the autoguider on delivery.

\section{Unpacking}

The following items should have been shipped:

\begin{itemize}
\item The FLI camera head.
\item The FLI camera head power supply, leads and USB cable.
\item The eeePC autoguider control computer.
\item The eeePC mounting plate and stand.
\item The eeePC power block and associated leads.
\end{itemize}

The eeePC comes pre-installed with the robotic autoguider software. A copy of the
code used to produce this is also included on the disk. Documentation will also be provided, either
electronically on disk and DVD, or physically on paper.

\section{Control computer}

For initial bench testing this will need to be connected to a monitor, keyboard and mouse. It should not be connected to the network initially.

\subsection{Power on}

The autoguider control computer is currently setup to boot when power is applied to it. So the computer should power up automatically. If it does not, there is a round power button on the front of the computer. Press it to power the machine up :- it lights with a blue LED when power is applied.

The computer should go through a standard Centos boot sequence, and eventually arrive at a command line login prompt. The computer is configured to {\em not} start a graphical user interface automatically, as this is not used for robotic operation.

\subsection{Login}

The usernames and passwords for the control computer are specified in the IUCAA software installation instructions \cite{bib:iucaasoftwareinstallation}, Section 4 (Users and Passwords). The {\bf eng} user is used for normal operations, but to change some system config files you need to be {\bf root}.

After logging in to the autoguider, type {\bf startx} to start a graphical user interface.

\subsection{Network configuration}

The network has been configured in the same way as the autoguider fitted to the Liverpool Telescope. This may {\bf not} be correct for the IUCAA telescope, and needs checking.

The Networks settings box can be displayed by selecting the  {\bf System \textgreater Preferences \textgreater Network connections} menu option, see Figure \ref{fig:networkconnections1}.


\setlength{\unitlength}{1in}
\begin{figure}[!h]
	\begin{center}
		\begin{picture}(6.0,5.0)(0.0,0.0)
			\put(0,0){\special{psfile=Network_Connections_1.eps   hscale=35 vscale=35}}
		\end{picture}
	\end{center}
	\caption{\em Network Connections Menu.}
	\label{fig:networkconnections1} 
\end{figure}

This brings up the {\bf Network Connections} dialog. By selecting {\bf System eth0} and pressing the {\bf Edit} button the {\bf Editing System eth0} dialog can be brought up as shown in Figure \ref{fig:networkconnections2}. This allows you to edit the IP Address, Netmask and Gateway. 

\setlength{\unitlength}{1in}
\begin{figure}[!h]
	\begin{center}
		\begin{picture}(6.0,5.0)(0.0,0.0)
			\put(0,0){\special{psfile=Network_Connections_2.eps   hscale=35 vscale=35}}
		\end{picture}
	\end{center}
	\caption{\em Editing System eth0.}
	\label{fig:networkconnections2} 
\end{figure}

On the Liverpool Telescope these need to be set as shown in Table \ref{tab:autogudiernetworksettings}.

\begin{table}[!h]
\begin{center}
\begin{tabular}{|l|p{20em}|}
\hline
{\bf Name} & {\bf Value}   \\ \hline
Address    & 192.168.1.3   \\ \hline
Netmask    & 255.255.255.0 \\ \hline
Gateway    & 192.168.1.254 \\ \hline
DNS Server & blank         \\ \hline
\end{tabular}
\end{center}
\caption{\em Autoguider network settings.}
\label{tab:autogudiernetworksettings}
\end{table}

On the IUCAA telescope they will need to be set as the old TTL autoguider was setup, so the TCS, SDB and MCP know
the location of the autoguider machine.

\subsubsection{/etc/hosts file}

The {\bf /etc/hosts} file contains the network addresses of machine the autoguider needs to communicate with. The file is setup as shown in the IUCAA software installation instructions \cite{bib:iucaasoftwareinstallation}, Section 8.2. The autoguider software needs the names {\bf mcc} , {\bf scc} and {\bf tcc} in the {\bf /etc/hosts} file, as these names are used in the autoguider configuration file and therefore looked up by the autoguider software when configuring software connections to the TCS (tcc), SDB (mcc), and MCP (scc). If the IP addresses differ on the IUCAA telescope they must be changed.

\subsection{Network testing}
\label{sec:networktesting}

When it is possible to connect the control computer to the telescope LAN the network settings can be tested. The following should be tested:

\begin{itemize}
\item From an autoguider terminal prompt, {\bf ping tcc} successfully transmits packets to and from the tcc. Here is an example of a successful ping:
\begin{verbatim}
[eng@acc ~]> ping tcc
PING tcc.lt.com (192.168.1.10) 56(84) bytes of data.
64 bytes from tcc.lt.com (192.168.1.10): icmp_seq=1 ttl=64 time=0.996 ms
64 bytes from tcc.lt.com (192.168.1.10): icmp_seq=2 ttl=64 time=0.244 ms
64 bytes from tcc.lt.com (192.168.1.10): icmp_seq=3 ttl=64 time=0.234 ms
64 bytes from tcc.lt.com (192.168.1.10): icmp_seq=4 ttl=64 time=0.238 ms

--- tcc.lt.com ping statistics ---
4 packets transmitted, 4 received, 0% packet loss, time 2999ms
rtt min/avg/max/mdev = 0.234/0.428/0.996/0.327 ms
\end{verbatim}
Use Ctrl-C to quit from the ping command.
\item From an autoguider terminal prompt, {\bf ping mcc} successfully transmits packets to and from the mcc.
\item From an autoguider terminal prompt, {\bf ping scc} successfully transmits packets to and from the scc.
\item Log into the mcc. From an mcc terminal prompt, {\bf ping acc} transmits packets to and from the autoguider. You may have to check {\bf acc} is setup correctly on the mcc, {\bf cat /etc/hosts} to find the autoguider entry. You may have to issue a {\bf slay ping} from a second mcc terminal to stop the ping, rather than using Ctrl-C.
\item From a machine on the telescope LAN, try {\bf ssh eng@acc} (if the machine has acc in the {\bf /etc/hosts}) or {\bf ssh eng@192.168.1.3} (replacing the IP address with the autoguider's if the IUCAA autoguider has a different IP address). You may have to use a different machine from the mcc for this test, it appears the Liverpool Telescopes ssh on mcc is {\bf not} compatible with the ssh on our autoguider machine. 
\end{itemize}

\section{Connecting the camera head}

To connect the camera head, The USB cable between the camera and control computer must be plugged in. The power supply brick can then be plugged into the camera head, and then into mains power and turned on.

If the control computer was turned on with the camera head plugged in and switched on, then it will automatically start the robotic software. If the camera head is plugged in after the control computer was switched on and fully booted, then the robotic software will have tried to start and failed (as no camera head was detected when the software was started).

You can tell if the robotic software is running by issuing the following command from a shell logged on to the autoguider control computer: {\bf ps waux | grep autoguider}. If the robotic software is running you will see something like the following:
\begin{verbatim}
[eng@acc ~]> ps waux | grep autoguider
root      2737  0.0  0.0   4532  1068 ?        S    10:36   0:00 /bin/sh -c ( 
cd /icc/bin/autoguider/c/i386-linux ; 
./autoguider -co autoguider.properties -autoguider_log_level 5 -ccd_log_level 5 
-command_server_log_level 5 -object_log_level 5 -ngatcil_log_level 5 1> 
/icc/log/autoguider_output.txt 2>&1 )
root      2738  0.0  0.0   4532   444 ?        S    10:36   0:00 /bin/sh -c ( 
cd /icc/bin/autoguider/c/i386-linux ; 
./autoguider -co autoguider.properties -autoguider_log_level 5 -ccd_log_level 5 
-command_server_log_level 5 -object_log_level 5 -ngatcil_log_level 5 1> 
/icc/log/autoguider_output.txt 2>&1 )
root      2744  0.1  0.1 101428  3268 ?        Sl   10:36   0:16 ./autoguider 
-co autoguider.properties -autoguider_log_level 5 -ccd_log_level 5 
-command_server_log_level 5 -object_log_level 5 -ngatcil_log_level 5
eng       6059  0.0  0.0   3924   704 pts/0    S+   14:53   0:00 grep autoguider
\end{verbatim}
whereas if the robotic software is not running you will see something like:
\begin{verbatim}
[eng@iucaaag ~]> ps waux | grep autoguider
eng       2646  0.0  0.0   4360   740 pts/0    S+   14:54   0:00 grep autoguider
\end{verbatim}

If the robotic software {\bf is} running, you need to stop it before running any of the low level test commands.

\section{Stopping the robotic software}

If the robotic software is running, it can be stopped by issuing the following commands:
\begin{verbatim}
su
(type in root password)
/icc/bin/autoguider/scripts/autoguider_engineering_mode
\end{verbatim}
The script will ensure the CCD was been warmed up before shutting down the robotic process.

\section{Low level command line testing}

The basic operation of the camera head can be tested with the following commands.

\subsection{test\_temperature\_low\_level}

This can be used to check the device driver is operating correctly, and that the control computer can communicate with the camera head correctly. It's main purpose is control and monitoring of the head CCD temperature.

The help option lists the command options:

\begin{verbatim}
[eng@iucaaag ~]> /icc/bin/autoguider/ccd/fli/test/i386-linux/test_temperature_low_level -help
Parsing Arguments.
Test Temperature (low level):Help.
test_temperature_low_level [-h[elp]] [-t[emperature] <target temp C>]
\end{verbatim}

Without any command line arguments the command will report some information about the camera head, and the current CCD temperature. Note because the device driver endpoint {\bf /dev/fliusb0} is owned by root, this command needs to be run as root (otherwise the command complains it can't find the camera head). Here is an example invocation:

\begin{verbatim}
[eng@iucaaag ~]> su
Password: 
[root@iucaaag eng]# /icc/bin/autoguider/ccd/fli/test/i386-linux/test_temperature_low_level
Parsing Arguments.
test_temperature_low_level: FLI library version: Software Development Library for Linux 1.104.
Find_FLI_Camera: Camera 0 has filename /dev/fliusb0 name MicroLine ML4720 and domain 258.
Find_FLI_Camera: Opening Camera with filename /dev/fliusb0 name MicroLine ML4720 and domain 258.
Camera Model: MicroLine ML4720
Serial Number: ML0832313
Hardware Revision: 256
Firmware Revision: 512
CCD Camera cold finger Temperature: 15.00.
CCD Temperature: 15.00.
Base Temperature: 24.19.
Cooler Power: 20.00.
\end{verbatim}

The camera cooler can be tested by turning on the cooler by specifying the set-point. This is done by using the -temperature argument:

\begin{verbatim}
[root@iucaaag eng]# /icc/bin/autoguider/ccd/fli/test/i386-linux/test_temperature_low_level 
-temperature -30
Parsing Arguments.
test_temperature_low_level: FLI library version: Software Development Library for Linux 1.104.
Find_FLI_Camera: Camera 0 has filename /dev/fliusb0 name MicroLine ML4720 and domain 258.
Find_FLI_Camera: Opening Camera with filename /dev/fliusb0 name MicroLine ML4720 and domain 258.
Camera Model: MicroLine ML4720
Serial Number: ML0832313
Hardware Revision: 256
Firmware Revision: 512
Setting temperature to -30.00 C
CCD Camera cold finger Temperature: 15.00.
CCD Temperature: 15.00.
Base Temperature: 24.25.
Cooler Power: 20.00.
\end{verbatim}

The temperature and cooler can then be monitored using this command, and the CCD can be warmed back to ambient by specifying a positive temperature.

\subsection{test\_exposure\_low\_level}

This command is a low level program for taking an exposure with the camera head and saving the resulting image to a FITS file. The help argument returns the following:

\begin{verbatim}
[root@iucaaag eng]# /icc/bin/autoguider/ccd/fli/test/i386-linux/test_exposure_low_level -help
Parsing Arguments.
Test Exposure (low level):Help.
test_exposure_low_level <-b[ias]|-d[ark] <exposure length ms>|-e[xpose] <exposure length ms>>
        -f[its_filename] <fits filename> [-h[elp]] -t[emperature] <target temp C>
        [-xb[in] <n>] [-yb[in] <n>].
        [-w[indow] <ulx> <uly> <lrx> <lry>].
The window is defined in unbinned pixels with upper left corner (<ulx>,<uly>) and lower right corner (<lrx>,<lry>).
\end{verbatim}

The command should be run as the root user, because the device driver endpoint {\bf /dev/fliusb0} is owned by root (otherwise the command complains it can't find the camera head).
As an example, a 1 second exposure may be taken as follows:

\begin{verbatim}
[root@iucaaag eng]# /icc/bin/autoguider/ccd/fli/test/i386-linux/test_exposure_low_level 
-expose 1000 -fits_filename test1.fits  -xbin 1 -ybin 1
Parsing Arguments.
test_exposure_low_level: FLI library version: Software Development Library for Linux 1.104.
Find_FLI_Camera: Camera 0 has filename /dev/fliusb0 name MicroLine ML4720 and domain 258.
Find_FLI_Camera: Opening Camera with filename /dev/fliusb0 name MicroLine ML4720 and domain 258.
Camera Model: MicroLine ML4720
Serial Number: ML0832313
Hardware Revision: 256
Firmware Revision: 512
Pixel Size (microns): 0.00001300 x 0.00001300
Array Area: Upper Left (0,0) Lower Right (1072,1033)
Visible Area: Upper Left (24,9) Lower Right (1048,1033)
CCD Camera cold finger Temperature: -30.00.
CCD Temperature: -30.00.
Base Temperature: 34.06.
Cooler Power: 95.00.
main: Allocating image data (1072 x 1033).
main: Setting image data to (0,0) (1072,1033).
main: Setting 1000 ms Exposure up.
main: Setting Binning to 1,1.
main: Starting exposure.
main: Waiting for the exposure to finish.
main: Camera status is WAITING_FOR_TRIGGER.
main: Camera status is READING_CCD.
main:Remaining exposure length = 1000 ms.
main: Camera status is EXPOSING.
main: Camera status is READING_CCD.
main:Remaining exposure length = 59 ms.
main: Camera status is IDLE.
main: Camera has data ready.
main:Remaining exposure length = 0 ms.
\end{verbatim}

In this example, the resultant exposure was saved in the test1.fits image file, which can be viewed using your favourite FITS image viewer. Both {\bf ds9} and {\bf gaia} are installed on the autoguider control computer, though only ds9 is available for the root user.

This test command has options to take both bias and dark images, but has the camera has no shutter the CCD should be covered with light-tight material to properly takes these images. With a pinhole placed in front of the CCD test images can be taken to check the camera head is imaging correctly.

\section{Starting the robotic software}

After using the low level command line tools, the robotic software can be restarted as follows:

\begin{itemize}
\item Command the CCD camera head to warm up to ambient:
\begin{verbatim}
/icc/bin/autoguider/ccd/fli/test/i386-linux/test_temperature_low_level 
-temperature 15
\end{verbatim}
\item Issue a command to check the CCD camera head temperature until it is warm enough to reboot the machine:
\begin{verbatim}
/icc/bin/autoguider/ccd/fli/test/i386-linux/test_temperature_low_level 
\end{verbatim}
\item When the camera head is warm enough, reboot the machine:
\begin{verbatim}
su
reboot
\end{verbatim}
\end{itemize}

If the camera head is on, the robotic software will be automatically started as part of the control computer boot-up procedure.

\section{Bench testing of the robotic software}

When the robotic software is running the following tasks should be considered before mounting the camera head on the telescope:

\begin{itemize}
\item Deciding on the CCD temperature set-point to use.
\item Creating a suitable set of dark frames for the temperature set-point.
\end{itemize}

\subsection{Temperature set-point}

During bench testing at the ARI we found the FLI camera head maintained a stable temperature when running at -30 degrees centigrade. However, with different ambient temperature conditions a different temperature set-point may be required. The best value to use is one where the camera head can maintain the temperature set-point over all the expected ambient conditions when mounted on the telescope.

The robotic software temperature set-point can be modified by editing the {\bf autoguider.properties} file located in the {\bf /icc/bin/autoguider/c/i386-linux/} directory on the autoguider control computer. The {\bf ccd.temperature.target } property is the one to change. See the Autoguider Configuration Guide \cite{bib:autoguiderconfigurationguide} for details. The robotic software will need to be restarted for this change to take effect.

\section{Setting up dark frames}

Whilst the autoguider is set up on the bench you may wish to take a set of dark frames. For the FLI camera head this requires the CCD to be covered (with black card or similar), as the head has no shutter. It may be easier to do this before mounting the camera head on the telescope. Section 10 (Calibration) in the Autoguider User Guide \cite{bib:autoguideruserguide} describes the process for doing this.

\section{Basic robotic imaging tests}
\label{sec:basicroboticimaging}

Before mounting the camera head on the telescope a basic test of the imaging capabilities of the robotic software may be useful. To do this a pinhole should be fitted to the front of the camera.

With the camera connected and the control computer booted, and running the robotic software, the following commands can be issued to take an image:

\begin{verbatim}
send_command -h iucaaag -p 6571 -c "expose 1000"
test_getfits_command -h iucaaag -p 6571 -c "getfits field raw" -f test1.fits
ds9 test1.fits &
\end{verbatim}

This tells the autoguider to take a 1 second exposure, which will be saved in the field buffer. The field buffer is then saved to the FITS filename test1.fits, and ds9 is then invoked to view the taken image. Obviously the exposure length and FITS filename can be changed as appropriate.

Once bench testing is complete the autoguider is ready to be mounted onto the telescope.


\section{Post telescope mounting tests}

Once the autoguider is mounted onto the telescope, and connected up and turned on, the following tests can be performed.
\begin{itemize}
\item The network testing in Section \ref{sec:networktesting} can be retested to ensure the network is working properly.\item Basic imaging of the camera head can be retested, see Section \ref{sec:basicroboticimaging}.
\end{itemize}

Before the autoguider can be used on-sky, the autoguider focus needs to be setup, and a suitable flat field frame acquired.

\subsection{Flat fielding}

The autoguider robotic software needs a suitable flat field to reduce field and guide images successfully These can be acquired at twilight or using dome flats. On the Liverpool Telescope we aim to get 10-30k counts in our flat field images, and find for dome flats this requires about 5 second exposures, and for sky flats 2 second exposures around 15 minutes after sunset. Given the IUCAA autoguider is using a different telescope and different camera head these values may need tuning.

The procedure used is described in the {\bf Autoguider User Guide}, Section 8.2 (Calibration, sub-section Flat). The robotic software will need to be restarted after the new flat is installed for the new flat to take effect.

\begin{thebibliography}{99}
\addcontentsline{toc}{section}{Bibliography}

\bibitem{bib:autoguideruserguide}{\bf Autoguider User Guide}
Liverpool John Moores University  \newline{\em /home/dev/public\_html/autoguider/latex/autoguider\_user\_guide.ps} v0.1

\bibitem{bib:autoguiderconfigurationguide}{\bf Autoguider Configuration Guide}
Liverpool John Moores University  \newline{\em /home/dev/public\_html/autoguider/latex/autoguider\_configuration\_guide.ps} v0.2

\bibitem{bib:iucaasoftwareinstallation}{\bf IUCAA autoguider control computer Operating System and Software Installation}
C. J. Mottram  \newline{\em /home/dev/public\_html/autoguider/latex/iucaaag\_software\_installation.ps} v0.1

\end{thebibliography}

\end{document}
